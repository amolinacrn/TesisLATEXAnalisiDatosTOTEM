\usepackage{amsmath}
\usepackage{chngcntr}
\counterwithout{footnote}{chapter}
\usepackage{endnotes}
%\usepackage{epsfig,bm,epsf,float}
%\setlength{\parskip}{-1.5ex}
\usepackage{mathrsfs}
\usepackage[latin1]{inputenc}
\usepackage{eqlist}
%\setlength{\parindent}{0cm} 
\usepackage{titlesec}
\usepackage{hyperref}
\usepackage[spanish,es-tabla]{babel}
\usepackage{lettrine}
%\usepackage{eso-pic}
\usepackage[left=4cm,top=2.5cm,right=2.5cm,bottom=2.5cm]{geometry}
\usepackage{graphicx}
\usepackage{fancybox}
\usepackage{mathrsfs}
\usepackage{amsfonts,amssymb,amsbsy}
\usepackage{anyfontsize}
\usepackage{fancyhdr}
\usepackage{multirow} % para las tablas
\usepackage{xcolor}
\usepackage{float}
\usepackage{upgreek} 
\usepackage{lmodern}
\usepackage{booktabs} % Horizontal rules in tables
\usepackage{relsize} 
\usepackage{eqparbox}
\renewcommand{\chaptermark}[1]{\markboth{\chaptername \thechapter. #1}{}}
\titleformat{\chapter}[display]
{\normalfont\Huge\bfseries}{\chaptertitlename\thechapter}{0pt}{\Huge}
\titlespacing*{\chapter}{0cm}{1cm}{0.8cm}
  \let\olditemize\itemize
\def\itemize{\olditemize\itemsep=-1ex }
%\renewcommand\thechapter{\large\arabic{chapter}}
\titleformat{\chapter}[display]
{\bfseries\Large}
{\large\rmfamily\bfseries\selectfont\color{gris}
\filleft\MakeUppercase{\chaptertitlename} \large\rmfamily\bfseries\thechapter}{0.8cm}
{\Huge\bfseries\selectfont\scshape\color{black}
\vspace{0ex}\filleft}
[{\titlerule\vspace{2pt}\titlerule[2.0pt]}]
%fin formato titulo

\definecolor{gris}{gray}{0.4}

%formato titulo

\renewcommand{\LettrineFontHook}{\color{amarilloM}}


\usepackage{titlesec, blindtext, color}  
%----------------------------------------------------------------------------------------
%	Numeración de las secciones -- en el margen
%----------------------------------------------------------------------------------------
 \definecolor{amarilloM}{gray}{0}%{RGB}{240,40,0}
 
\makeatletter
\renewcommand{\@seccntformat}[1]{\llap{\textcolor{gris}{\csname the#1\endcsname}\hspace{1em}}}                    
\renewcommand{\section}{\@startsection{section}{1}{\z@}
{-4ex \@plus -1ex \@minus -.4ex}
{1ex \@plus.2ex }
{\Large\sffamily\bfseries\selectfont\color{amarilloM}}}
\renewcommand{\subsection}{\@startsection {subsection}{2}{\z@}
{-3ex \@plus -0.1ex \@minus -.4ex}
{0.5ex \@plus.2ex }
{\Large\sffamily\bfseries\selectfont\color{amarilloM}}}
\renewcommand{\subsubsection}{\@startsection {subsubsection}{3}{\z@}
{-2ex \@plus -0.1ex \@minus -.2ex}
{0.2ex \@plus.2ex }
{\normalfont\small\sffamily\bfseries}}                        
\renewcommand\paragraph{\@startsection{paragraph}{4}{\z@}
{-2ex \@plus-.2ex \@minus .2ex}
{0.1ex}
{\normalfont\small\sffamily\bfseries}} 
 \makeatletter
 
 
 
\renewcommand{\@seccntformat}[1]{{\textcolor{gris}{\csname the#1\endcsname}\hspace{1em}}} 
                   
\renewcommand{\section}{\@startsection{section}{1}{\z@}
{0.5cm \@plus 0.1cm \@minus 0.1cm}
{0.4cm \@plus 0.1cm \@minus 0.1cm }
{\Large\sffamily\bfseries\selectfont\color{amarilloM}}}

\renewcommand{\subsection}{\@startsection {subsection}{2}{\z@}
{0.5cm \@plus 0.1cm \@minus 0.1cm}
{0.4cm \@plus 0.1cm \@minus 0.1cm }
{\large\sffamily\bfseries\selectfont\color{amarilloM}}}

\renewcommand{\subsubsection}{\@startsection {subsubsection}{3}{\z@}
{0.1cmm \@plus -0.1ex \@minus -.2ex}
{0.03 \@plus.2ex }
{\normalfont\small\sffamily\bfseries\selectfont\color{amarilloM}}}                        

\renewcommand\paragraph{\@startsection{paragraph}{4}{\z@}
{-2ex \@plus-.2ex \@minus .2ex}
{0.1ex}
{\normalfont\small\sffamily\bfseries\selectfont\color{amarilloM}}}
\makeatother

% Fin numeración secciones
\fancyhf{}
%% Now begin customising things. See the fancyhdr docs for more info.
\usepackage{titlesec, blindtext, color} 

\fancyhead[LE]{{\footnotesize\sffamily\bfseries\sffamily \thepage}
\scriptsize \sffamily{\hspace{0.08cm} \leftmark}}
%\setcounter{tocdepth}{2}
%\fancyhead[LE]{}
%\fancyhead[RE,LO]{}
\fancyhead[RO]{
%{\footnotesize\sffamily\thesection}\hspace{0.08cm}
{\footnotesize \sffamily \rightmark}\hspace{0.3cm}
{\footnotesize\sffamily\bfseries\thepage}}

\renewcommand{\sectionmark}[1]{\markright{\sffamily{\thesection\ \ #1}}}
\renewcommand{\chaptermark}[1]{\markboth{{\scriptsize\sffamily CAP\'ITULO\ \thechapter}  \scriptsize\sffamily{ \ #1}}{}} 

%%barras para letras

\makeatletter
\newsavebox\myboxA
\newsavebox\myboxB
\newlength\mylenA

\newcommand*\xoverline[2][0.6]{%
    \sbox{\myboxA}{$\m@th#2$}%
    \setbox\myboxB\null% Phantom box
    \ht\myboxB=\ht\myboxA%
    \dp\myboxB=\dp\myboxA%
    \wd\myboxB=#1\wd\myboxA% Scale phantom
    \sbox\myboxB{$\m@th\overline{\copy\myboxB}$}%  Overlined phantom
    \setlength\mylenA{\the\wd\myboxA}%   calc width diff
    \addtolength\mylenA{-\the\wd\myboxB}%
    \ifdim\wd\myboxB<\wd\myboxA%
       \rlap{\hskip 0.8\mylenA\usebox\myboxB}{\usebox\myboxA}%
    \else
        \hskip -0.5\mylenA\rlap{\usebox\myboxA}{\hskip 0.5\mylenA\usebox\myboxB}%
    \fi}
\makeatother

\fancyfoot{}
\usepackage{color}
\definecolor{gray97}{gray}{0.95}
\definecolor{gray75}{gray}{.65}
\definecolor{gray45}{gray}{.45}

\usepackage{listings}
\lstset{ frame=Ltb,
framerule=0pt,
aboveskip=0.5cm,
framextopmargin=3pt,
framexbottommargin=3pt,
framexleftmargin=0cm,
framesep=0pt,
rulesep=.4pt,
backgroundcolor=\color{gray97},
rulesepcolor=\color{black},
%
stringstyle=\ttfamily,
showstringspaces = false,
basicstyle=\small\ttfamily,
commentstyle=\color{gray45},
keywordstyle=\bfseries,
%
numbers=left,
numbersep=0.1cm,
numberstyle=\tiny,
numberfirstline = false,
breaklines=true,
}

% minimizar fragmentado de listados
\lstnewenvironment{listing}[1][]
{\lstset{#1}\pagebreak[0]}{\pagebreak[0]}

\lstdefinestyle{consola}
{basicstyle=\scriptsize\bf\ttfamily,
backgroundcolor=\color{gray75},
}

\lstdefinestyle{C}{
  language=C
}

