 % -----------------------------------------------------------------------------------------------------------------
% Portada
% Cambiar los nombres...
%-----------------------------------------------------------------------------------------------------------------

% Nombre de tesis
%\centering
\title{Lagrangianos de orden superior en teor�a cl�sica de campos} 

% Autor
\author{\sffamily Miguel Alejandro Molina}
% Mes
\degreemonth{Abril}

% A�o
\degreeyear{2013}

% T�tulo profesional
\degree{F�sico}

% Nombre de la tesis en ingl�s
\engtitle{Classical field theory with higher \mbox{derivatives}} 


\maketitle%NO BORRAR

% -----------------------------------------------------------------------------------------------------------------
% Contraportada
% Cambiar los nombres...
%-----------------------------------------------------------------------------------------------------------------
% Asesor
\director{Ph.D. Carlos Avila}

\maketitled%NO BORRAR
%-------------------------------------

\copyrightpage%NO BORRAR

\aceptacion%NO BORRAR

% #####################################################
% OBS: Se modifican la parte de las 'xxx xxx xxxxx'   #
% #####################################################
% -----------------------------------------------------------------------------------------------------------------
% Cambiar dedicatoria...
% -----------------------------------------------------------------------------------------------------------------
\dedication



% -----------------------------------------------------------------------------------------------------------------
% Cambiar agradecimientos...
% -----------------------------------------------------------------------------------------------------------------

\begin{acknowledgments}
Agradezco al departamento de f�sica de la universidad de Nari�o por darme la \mbox{oportunidad} de cumplir mi ideal, ser F�sico. Un agradecimiento especial a mi asesor de trabajo de grado el Ph.D. Carlos Arturo Avila Bernal, profesor titular de la Universidad de los andes. Por su gran compromiso, su tiempo y por compartir sus conocimientos.
\\

A mi familia, especialmente a mi madre Maria Ceron por darme la vida, a mis padrinos: Jes�s Ortiz y Gerardina Gutierrez, a su hijo Pablo Ortiz, a mi hermana Yaqueline Molina. Por su amor, su respaldo incondicional. Todo lo que soy es gracias a ustedes. 
\\

A mis amigos y compa�eros de f�sica, especialmente a Ivan Romo, al profesor Jaime Betancurth, a todos y cada uno de ustedes,\\
\begin{center}
 \huge{GRACIAS}. 
\end{center}

\end{acknowledgments}




% -----------------------------------------------------------------------------------------------------------------
% Cambiar resumen en espa�ol...
% -----------------------------------------------------------------------------------------------------------------
\begin{abstract}
\begin{itshape} 
\dsp
En este trabajo se estudiar� una teor�a cl�sica de campos con Lagrangianos de orden \mbox{superior} en las derivadas. Los principios b�sicos de una teor�a cl�sica de campos \mbox{permitir�} \mbox{generalizar} las ecuaciones de campo, las ecuaciones de Hamilton y el primer teorema de Noether asociado con el grupo de Poincar�. Se deducir� el segundo teorema de Noether para teor�as gauge con derivadas superiores, el estudio can�nico de estas teor�as se realizar� por el m�todo de Dirac. Por �ltimo, se estudia la teor�a electromagn�tica de Podolsky.
\end{itshape}
\end{abstract}


% -----------------------------------------------------------------------------------------------------------------
% Cambiar resumen en ingl�s...
% -----------------------------------------------------------------------------------------------------------------

\begin{engabstract}
\begin{itshape}
In this work we are going to study classical field theory with higher derivatives. The basic principles of classical field theory will allow to generalize the field equations, Hamilton's equations and the first Noether theorem associate with the group of Poincar�. Here, we will deduce the second Noether theorem for gauge theories describe by higher derivatives and its canonical study will be realized by the Dirac method. Finally, we are going to study the Podolsky electromagnetic theory.
\end{itshape}
\end{engabstract}




%\newpage
%\addcontentsline{toc}{section}{Tabla de contenidos}

\tableofcontents

\setlength{\parskip}{-1.5ex}

%\newpage
% Generates the List of Symbols. The argument should point to
% the file containing your List of Symbols. 
\thesislistofsymbols{Glosario/glosario}
%\listofappendix


\newpage

\startarabicpagination
