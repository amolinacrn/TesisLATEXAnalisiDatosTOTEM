 % -----------------------------------------------------------------------------------------------------------------
% Portada
% Cambiar los nombres...
%-----------------------------------------------------------------------------------------------------------------

% Nombre de tesis
%\centering

\title{Estudio de la secci�n eficaz diferencial el�stica para colisiones prot�n (antiprot�n)-prot�n basada en un modelo quark-diquark con elasticidad del pomer�n} 

% Autor
\author{\sffamily  Miguel Alejandro Molina}
% Mes
\degreemonth{Enero}

% A�o
\degreeyear{2017}

% T�tulo profesional
\degree{F�sico}

% Nombre de la tesis en ingl�s
\engtitle{Study of the proton (antiproton)-proton elastic differential cross section based on a quark-diquark model with pomeron elasticity.} 


\maketitle%NO BORRAR

% -----------------------------------------------------------------------------------------------------------------
% Contraportada
% Cambiar los nombres...
%-----------------------------------------------------------------------------------------------------------------
% Asesor
\director{Ph.D. Carlos Arturo \'Avila Bernal}
%\codirector{}
\maketitled%NO BORRAR
%-------------------------------------

\copyrightpage%NO BORRAR

\aceptacion%NO BORRAR

% #####################################################
% OBS: Se modifican la parte de las 'xxx xxx xxxxx'   #
% #####################################################
% -----------------------------------------------------------------------------------------------------------------
% Cambiar dedicatoria...
% -----------------------------------------------------------------------------------------------------------------
\dedication

\begin{quote}
\hsp \em \raggedleft


\end{quote}


% -----------------------------------------------------------------------------------------------------------------
% Cambiar agradecimientos...
% -----------------------------------------------------------------------------------------------------------------
\begin{acknowledgments}
Agradezco al departamento de f�sica de la universidad de Nari�o por darme la \mbox{oportunidad} de cumplir mi ideal, ser F�sico. Un agradecimiento especial a mi asesor de trabajo de grado el Ph.D. Carlos Arturo \'Avila Bernal, profesor titular de la Universidad de los Andes: por su gran compromiso,  por compartir su tiempo y conocimientos. A mi familia, amigos, compa�eros y docentes del programa de f�sica, especialmente al profesor Jaime Betancourt.
\end{acknowledgments}





% -----------------------------------------------------------------------------------------------------------------
% Cambiar resumen en espa�ol...
% -----------------------------------------------------------------------------------------------------------------
\begin{abstract}
\begin{itshape} 
\dsp
Se presenta el estudio de un modelo de dispersi�n el�stica de nucleones (y anti-nucleones) basado  en una representaci�n quark-diquark $(qQ)$  del nucle�n con pomer�n el�stico. Este modelo aumenta la parte real de  la amplitud de dispersi�n mejorando su descripci�n en el m�nimo de difracci�n. Las predicciones del modelo se comparan con los datos experimentales disponibles para cada una de las secciones eficaces diferenciales el�sticas  de los nucleones en  un rango de energ�a entre 4.26 GeV hasta 7 TeV. Esta parametrizaci�n no describe correctamente los  datos experimentales para ciertas energ�as en colisiones prot\'on-prot\'on (y antiprot\'on-prot\'on). Se incluye un estudio que trata de mejorar el modelo de ajuste a  los datos existentes, presentando buenos resultados.
\end{itshape}
\end{abstract}


% -----------------------------------------------------------------------------------------------------------------
% Cambiar resumen en ingl�s...
% -----------------------------------------------------------------------------------------------------------------

\begin{engabstract}
\begin{itshape}
We present a study of the nucleon-nucleon (an nucleon-antinucleon) elastic differential cross section  based on a representation quark-diquark (qQ) of the nucleon with elastic pomeron. This model increases the real part of the scattering amplitude improving the description of the diffraction minimum. The predictions of the model are compared to the available experimental data within the range of 4.26 GeV through 7 TeV. This parametrization does not describe correctly the experimental data of proton-proton (proton-antiproton) collissions for several energies. We include a study aiming to improve the model description to existing data, obtaining good results.
\end{itshape}
\end{engabstract}




%\newpage
%\addcontentsline{toc}{section}{Tabla de contenidos}

\tableofcontents

\begin{listoffigures}{}
\end{listoffigures}

\begin{listoftables}{}
\end{listoftables}

\thesislistofsymbols{Glosario/glosario}
\setlength{\parskip}{-1.5ex}
\newpage
\startarabicpagination

%\newpage
% Generates the List of Symbols. The argument should point to
% the file containing your List of Symbols. 
%\thesislistofsymbols{Glosario/glosario}
%\listofappendix


%\newpage

%\startarabicpagination
